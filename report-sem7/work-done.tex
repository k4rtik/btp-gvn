\chapter{Work Done}

\section{Previous Work (Monsoon Semester 2012)}

In the first phase of the project we started out with basic literature survey on Global Value Numbering to solidify essential concepts and study the approaches taken by researchers in the past.

Next, a study of available options among compiler infrastructures was done and GCC\cite{gcc} identified as the implementation platform of choice owing to its open-source nature, industry recognition and popularity.

A high level study of structure of GCC, including its intermediate representation (IR) forms (viz. GENERIC\cite{generic}, GIMPLE\cite{gimple} and RTL\cite{rtl}) and pipeline flow in GCC was done next.

Lastly, some practical experience was gained with a na\"{\i}ve implementation of constant propagation optimization as a GCC plugin.

\section{Progress}

\subsection{Choice of Algorithm for implementation}

We have identified \textbf{A Simple Algorithm for Global Value Numbering} by \textit{N Saleena, Vineeth Paleri} as the algorithm of choice for implementation of GVN on GCC framework. Salient features of this new algorithm include its complete nature in terms of number of identifiable redundancies (same as that of Kildall's approach), introduction of the concept of a \textit{Value Expression} to represent a set of equivalent expressions, and the inherent simplicity of the algorithm.

\subsection{Performance Comparison}

To compare the efficiency and performance of proposed implementation of the new algorithm, it was necessary to identify few good algorithms and benchmarks to run the tests.

We have identified the following two algorithms for comparison, both of which are already implemented in GCC:
\begin{itemize}
	\item SCC-Based Value Numbering\cite{scc-vn} (\texttt{tree-ssa-sccvn.c} in GCC source) -- This is the current implementation of value numbering in GCC.
	\item GVN-PRE\cite{vandrunen,gvn-pre} (\texttt{tree-ssa-pre.c} in GCC source) -- This is a value based Partial Redundancy Elimination algorithm, which looks like a good choice to observe the difference in number of redundancies identified by our implementation which does not include partial redundancies.
\end{itemize}

We have also identified SPEC CPU2006\cite{spec-cpu06} as the benchmark suite for comparison testing based on previous work done in this area. More clarity on which tests among the suite are suitable for GVN is expected to be gained after the implementation.

\subsection{Study of GCC APIs for Pass Implementation}

We have chosen GCC version 4.6.3 for implementation.

A comprehensive study of GCC Application Programming Interface for implementation of optimization passes is under progress using GCC source code, GCC Internals Documentation\cite{gccint}, and slides from GCC Internals Course\cite{internals-course}, which is necessary to proceed with the proposed implementation.
